\documentclass[10pt,twocolumn,letterpaper]{article}

\usepackage{cvpr}
\usepackage{times}
\usepackage{epsfig}
\usepackage{graphicx}
\usepackage{amsmath}
\usepackage{amssymb}
% Include other packages here, before hyperref.

% If you comment hyperref and then uncomment it, you should delete
% egpaper.aux before re-running latex.  (Or just hit 'q' on the first latex
% run, let it finish, and you should be clear).
\usepackage[pagebackref=true,breaklinks=true,letterpaper=true,colorlinks,bookmarks=false]{hyperref}
%%%%%%%%% PAPER ID  - PLEASE UPDATE
\def\cvprPaperID{7466} % *** Enter the CVPR Paper ID here
\def\httilde{\mbox{\tt\raisebox{-.5ex}{\symbol{126}}}}

\newcommand{\Paragraph}[1]{\vspace{0.40mm} \noindent \textbf{#1} \hspace{0mm}}

\begin{document}

%%%%%%%%% TITLE - PLEASE UPDATE
\title{The Edge of Depth: Explicit Constraints between Segmentation and Depth}  % **** Enter the paper title here

\maketitle
\thispagestyle{empty}


%%%%%%%%% BODY TEXT - ENTER YOUR RESPONSE BELOW
\noindent We sincerely thank reviewers for comments \& suggestions.
All reviewers agree on the novelty and thoroughness of experiments. 
We address concerns and suggestions as follows:
%\vspace{-5.5mm}
%\paragraph*{\textcolor{blue}{Reviewer1:}}

\Paragraph{\textcolor{blue}{R1:}}
\textbf{Bleeding artifacts.} 
% We reason occlusion across stereo images is causation of bleeding artifacts. Since stereo occlusion only happens horizontally, theoretically it can be entirely handled by left-righfdot mask. 
We find horizontal occlusion across stereo images as the causation of bleeding artifacts. 
From this point, in theory the left-right mask can handle the artifacts entirely.
We consider the ``vertical" artifacts stem from a general ambiguity, but are improved by $l_g$ loss which corrects most of such border mismatches between segmentation \& depth. 
\textbf{Suppl.} $4$ first subfigure shows an example where a cyclist's fading head is recovered after imposing $l_g$. 
Further, although mask $\mathbf{M}$ is computed horizontally, it still detects on the top / bottom of an object as in Fig $4$.c.

\Paragraph{Weight normalization.} 
In our design, $w(\cdot)$ controls mutual influence among edge pairs while $h(\cdot)$ controls a single pair's effective range, non-normalized to be independent of the number of pairs. 
Consider when only one pair exists, if both terms are normalized then each pixel within the image will incorrectly receive the same shift.
We will emphasize this intuition further in the paper (Sec.~$3.1.2$). 

\Paragraph{Additional ResNet18 entry.} 
As suggested, we will add a ResNet18 entry in Tab. $2$ and Fig. $5$. for fair comparison.

\Paragraph{Code Release.} 
We look forward to releasing our training/testing code to enable related research and future work. 

%\Paragraph{Reference to supplementary.} 
%We agree and will add references to the supplementary in relevant parts of the paper.

\Paragraph{\textcolor{blue}{R2:~}}
\textbf{$l_g$ within occluded region.}
The loss defined in Eqn. $13$ is correct such that no supervision signal is imposed in occluded regions. 
While using $l_g$ we face multiple noise sources including from depth border, semantics border and the morphing process itself. %(the two borders not necessarily overlap). 
%Although the $l_g$ term may sometimes perform better than photometric in occluded areas it suffers from noise when presented as the only constraint. % to noise from the depth border, semantics border and the morphing process itself.
Due to this noise, the photometric loss and $l_g$ terms are most benefited when constrained together in unoccluded regions. 
%To relieve the noise, we constrain the photometric loss and $l_g$ terms by each other, imposing $l_g$ on unoccluded area. 
Further, we notice that where the photometric loss fails to provide enough constraints, such as near frequent occluded regions around sides of objects or low-textures, the noise inherited from $l_g$ will lead to artifacts as depicted in Fig. $8$ Col. (d).
Such effects will be more severe if we impose $l_g$ in occluded area.
% The low-textureness and frequent occlusion make image area around two sides of foreground objects fail to provide strong constrain. 
% If we impose $l_g$ around occluded area, the artifacts in column (d) will be even worse. 
However, this strategy will not discount the functionality of $l_g$ on occluded left borders since the left stereo image is still supervised from right stereo image's unoccluded left border (\& vice versa for the right side). 
%In conclusion, this strategy trades 50\% similar training material for higher quality supervision signal, which is acceptable for a self-supervised framework and circumvent dilemma pointed out.
% To resolve that, we adopt $l_p$ to stabilize training, as shown in column (e).
%In conclusion, this strategy filters out redundant low quality training data, circumventing the dilemma pointed out.

\Paragraph{Baseline selection.} 
We agree that comparing experiments against Baseline $ +~{l_p}$ is more fair in Fig. $5\text{-}7$, $9$ and will update accordingly. 
For purposes of ablation, we select $[11]$ as baseline to show the importance of $l_p$ for stabilizing $l_g$. 
Since self-supervised depth typically forms joint loss with imperfect modalities, robustness is increasing critical. %, whereas [38] does not discuss this specific benefit. 
We find ${l_p}$ is effective to remedy weak photometric supervision near object boundaries since they introduce additional supervision from comparatively high quality SGM predictions. 
We demonstrate the effect in Fig. $8$ Col. (d) \& (e) and \textbf{Suppl.} $4$ right column. 
As such, we will clarify the related description further in lines $511\text{-}518$ and $747\text{-}755$.
%For Table $5$, it's better conducted on public available model. \\

\Paragraph{Small improvements in Table 3.} 
We emphasize Tab. $3$ measures the improvement between an off-the-shelf prediction and its morphed result without retraining the models. 
It is expected to be small as it stands for single step improvement under a greedy optimization framework. 
We cannot repeatedly apply the morph since the border gradient would be reduced to $\frac{t}{t+1}$ (\textbf{Suppl.} Eqn. $3$).
%Small improvement in Tab. $3$ is due to local optimality of greedy optimization, rather than their prediction's quality. 
Hence, Tab.~$3$ serves as a proof-of-concept showing that supervised methods predict imperfect borders and may benefit from $l_g$ in future work. 
%Under this perspective, we can not say those models possess good borders.

\Paragraph{Inferior ${l_p}$ performance.} 
Hyperparameters such as resolution and learning rate will have an impact on performance, but we do not consider as the dominant reasons. 
We suspect it is more related to inferior SGM quality. 
For instance, [$38$] uses combinations of three block sizes with four disparities resolutions to produce SGM predictions. 
However, since $[38]$ did not publish the detailed hyperparameters, it is time-consuming to search for the optimal combination. 
Further, only the last entry in Tab. $4$, entitled ``Finetune", utilizes ``$3$ epoch fine-tuning". 
We expect this strategy may also benefit $l_p$ if ablated, although it is not used in [$38$]. %does not use this strategy, although if further ablated. 

\Paragraph{Benefits for other tasks.} 
It is an excellent suggestion for future exploration, currently beyond the scope of our paper. 
%We also notice performance can be further improved, as stated in line $645$ - $675$, however within the scope of our paper, we wish to be more focused on our topic, "border consistence". 
We plan to publish our code to benefit other research in exploring such directions including more tasks or modalities.

\Paragraph{Computation of $\text{Var}(\mathbf{I})$. } 
The $\text{Var}(\mathbf{I})$ is computed within a $3 \times 3$ local window, as an average across RGB channels, which will be added to implementation details (Sec.~$3.3.2$).

%\vspace{-6.5mm}
\Paragraph{\textcolor{blue}{R3:}}
\Paragraph{Performance gain.} 
Although the improvement is not dramatic, our relative performance gain is comparable with recent works.
For instance, we emphasize that our method reduces the gap between SOTA supervised and unsupervised methods by $60\%$ (line $590$), similar to $64.29\%$ in [$38$], $59.46\%$ in [$11$], $38.64\%$ in [$3$] and $42.86\%$ in [$33$]. 

\Paragraph{Paper layout \& writing.} 
We will add more overview of our pipeline, including functionality of each loss term, to the beginning of Sec.~$3$. 
The contents of Fig.~$2$ will be improved with its caption modified to ensure the input \& output are directly stated with contrast to supervised methods. 

\Paragraph{Overload in $\mathbf{I_s}$.} 
Thank you for pointing out the inconsistency. 
We mistakenly use $\mathbf{I_s}$ to refer to a semantic segmentation map in line $265$ and Fig. $2$, which we will correct. 
%Since $\mathbf{I_s}$ is a binary segmentation map (defined in line $294$), we will correct the consistencies in both text and figures for the final version. 
$\mathbf{I_s}$ is defined (line $294$ \& $538$) as a binarized segmentation map derived from Cityscapes~[5] $19$ semantic categories. 
The edge of the segmentation is denoted as $\mathbf{T}_s$ in Eqn.~$1$
%Foreground includes Cityscape object, vehicle, and human categories, listed in line $538$.

\Paragraph{More illustration in supplementary.} 
We agree and will follow this suggestion by adding additional illustrations of the terms $\mathbf{T_s}$, $\mathbf{T_d}$, $\mathbf{\delta}$, $\mathbf{\Gamma}$ in the supplementary material.

\end{document}
